\documentclass[]{article}
\usepackage{amsmath}
\usepackage{pdflscape}
\usepackage{booktabs}
\usepackage{array}
\usepackage{threeparttable}
\usepackage{fancyhdr}
\usepackage{lastpage}
\usepackage{textcomp}
\usepackage{dcolumn}
\newcolumntype{L}[1]{>{\raggedright\let\newline\\\arraybackslash\hspace{0pt}}m{#1}}
\newcolumntype{C}[1]{>{\centering\let\newline\\\arraybackslash\hspace{0pt}}m{#1}}
\newcolumntype{R}[1]{>{\raggedleft\let\newline\\\arraybackslash\hspace{0pt}}m{#1}}
\newcolumntype{.}{D{.}{.}{-1}}
\usepackage[T1]{fontenc}
\usepackage{caption}
\usepackage{subcaption}
\usepackage{graphicx}
\usepackage[margin=0.8in, bottom=1.2in]{geometry}
\usepackage[page]{appendix}
\pagestyle{fancy}
\renewcommand{\headrulewidth}{0pt}
\fancyhead{}
\rfoot{Page \thepage\  of \pageref{LastPage}}
\cfoot{}
\begin{document}
\title{Combining Excel and Python}
\author{Nick DeRobertis}
\date{\today}
\maketitle
\begin{section}{Introduction to Combining Excel and Python}
\begin{itemize}
\item At this point in the course, you should feel comfortable using both Python and Excel to create models to solve problems
\item Now it is time to learn how to combine the two tools for the maximum flexibility, power, and convenience
\item We will cover two approaches to integrating the two: using Pandas and using xlwings
\item We are also about to learn Monte Carlo simulation, which can be done easily in Python but would require using VBA or an extension in Excel. Using this combination we can have the model in Excel and run Monte Carlo simulations on it in Python
\item The Pandas approach is simpler but is much more limited, basically you can read in Excel workbooks and you can output an entire workbook or sheet
\item The xlwings approach gets a bit more complicated but allows to have a connection between Excel and Python and transfer individual values or entire tables back and forth with an existing workbook
\end{itemize}
\end{section}
\begin{section}{Combining Excel and Python using Pandas}
\begin{itemize}
\item If you have a Python model and you just want to load some data in from Excel, Pandas is probably your best choice
\item If you have an Excel model and you collect the data using Python, this is also a good choice
\item If you want to have some parts of your model in Excel and some parts in Python, you should probably look to xlwings
\item BE CAREFUL WHEN WRITING TO WORKBOOKS as it will replace what is there. It could overwrite your Excel model. THERE IS NO UNDO (back up your work)
\end{itemize}
\end{section}
\begin{section}{Combining Excel and Python using xlwings}
\begin{itemize}
\item xlwings is a package that makes it quite easy to combine Excel and Python in ways that should work for nearly every use case
\item This can be done without xlwings using the Microsoft COM API in Python, but xlwings is far more convenient
\item We are focusing here on only manipulating Excel from Python, but I encourage you to explore running Python from Excel on your own time
\item Now with xlwings, anything that you can do in Excel, you can make it happen from Python
\item It is easy to transfer individual values, ranges, and tables back and forth between Excel and Python
\item For those who have a lot of difficulty with Python and feel comfortable in Excel, xlwings allows building out the core model in Excel and adding extensions such as Monte Carlo simulation, sensitivity analysis, and scenario analysis in Python
\end{itemize}
\end{section}
\end{document}