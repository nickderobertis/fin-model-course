\documentclass[]{article}
\usepackage{amsmath}
\usepackage{pdflscape}
\usepackage{booktabs}
\usepackage{array}
\usepackage{threeparttable}
\usepackage{fancyhdr}
\usepackage{lastpage}
\usepackage{textcomp}
\usepackage{dcolumn}
\newcolumntype{L}[1]{>{\raggedright\let\newline\\\arraybackslash\hspace{0pt}}m{#1}}
\newcolumntype{C}[1]{>{\centering\let\newline\\\arraybackslash\hspace{0pt}}m{#1}}
\newcolumntype{R}[1]{>{\raggedleft\let\newline\\\arraybackslash\hspace{0pt}}m{#1}}
\newcolumntype{.}{D{.}{.}{-1}}
\usepackage[T1]{fontenc}
\usepackage{caption}
\usepackage{subcaption}
\usepackage{graphicx}
\usepackage[margin=0.8in, bottom=1.2in]{geometry}
\usepackage[page]{appendix}
\pagestyle{fancy}
\renewcommand{\headrulewidth}{0pt}
\fancyhead{}
\rfoot{Page \thepage\  of \pageref{LastPage}}
\cfoot{}
\begin{document}
\title{The Depth of a Financial Model, Continued}
\author{Nick DeRobertis}
\date{\today}
\maketitle
\begin{section}{Using Jupyter to Structure a Python Model}
\begin{itemize}
\item It can be a bit tricky in the beginning to structure Python models in Jupyter as you are dealing with two different layers of organization
\item Jupyter gives us nicely formatted markdown cells which make it easy to organize sections of the model
\item Markdown is actually a general markup language, not anything specific to Jupyter, and it supports a lot of features. Jupyter has their own extension to markdown which also adds LaTeX equation support
\item Most often, you will just need section headers, bullets, and equations, and anything else you can look at a Markdown reference
\item It is easy to add a table of contents for a Jupyter notebook and you should do this to increase the readability of your model
\item When adding a TOC item, spaces get converted to dashes for the reference
\end{itemize}
\end{section}
\begin{section}{Salaries in the Python Dynamic Salary Retirement Model}
\begin{itemize}
\item For development purposes, create a new variable data which is set equal to model\_data. When you are done with the model, you will remove this.
\item Write the logic for a function in a cell and run it to ensure it works, then move it into a function
\item Using data in the functions while the original variable is model\_data ensures that you are not accidentally accessing the overall (global) model\_data when it should be the specific instance of ModelInputs being passed
\item This may be confusing and sound like extra unnecessary steps, but setting things up this way will enable your model to be easily extended
\item Here we will create a function which can get the salary in any given year
\item We will write also some example code to test the function and show its results
\item Later we will use this function in the overall calculation
\end{itemize}
\end{section}
\begin{section}{Wealth in the Python Dynamic Salary Retirement Model}
\begin{itemize}
\item Here we will develop two functions which comprise the wealth sub-model
\item First create a function which determines the amount of cash saved in a given year
\item Then create a function which determines the amount of wealth in a given year
\item We create some example code to show how the function works, but it will actually be applied in the Retirement sub-model
\end{itemize}
\end{section}
\begin{section}{Retirement in the Python Dynamic Salary Retirement Model}
\begin{itemize}
\item Now we will bring everything together to calculate the years to retirement
\item The salary and cash saved functions are already getting called from within the wealth function, so we only need to call the wealth function in the final loop
\item Here we are making use of a while loop to stop the loop once a certain condition is met, in this case once wealth exceeds desired cash
\item We will use formatted strings and new lines to create a good display for the output
\end{itemize}
\end{section}
\begin{section}{Lab Exercise}
\begin{itemize}
\item Feel free to work from the example model though I would recommend you build that out yourself following the prior videos
\item This exercise is exactly the same as the one we did for Excel to calculate the desired cash rather than taking it as an input
\item Hint: You should add the new inputs to the ModelInputs dataclass and remove the desired cash input. Then you can create a function which calculates the desired cash based on the model inputs, and use that in place of where the desired cash was being accessed directly before
\end{itemize}
\end{section}
\end{document}