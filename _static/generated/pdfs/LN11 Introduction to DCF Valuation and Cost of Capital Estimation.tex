\documentclass[]{article}
\usepackage{amsmath}
\usepackage{pdflscape}
\usepackage{booktabs}
\usepackage{array}
\usepackage{threeparttable}
\usepackage{fancyhdr}
\usepackage{lastpage}
\usepackage{textcomp}
\usepackage{dcolumn}
\newcolumntype{L}[1]{>{\raggedright\let\newline\\\arraybackslash\hspace{0pt}}m{#1}}
\newcolumntype{C}[1]{>{\centering\let\newline\\\arraybackslash\hspace{0pt}}m{#1}}
\newcolumntype{R}[1]{>{\raggedleft\let\newline\\\arraybackslash\hspace{0pt}}m{#1}}
\newcolumntype{.}{D{.}{.}{-1}}
\usepackage[T1]{fontenc}
\usepackage{caption}
\usepackage{subcaption}
\usepackage{graphicx}
\usepackage[margin=0.8in, bottom=1.2in]{geometry}
\usepackage[page]{appendix}
\pagestyle{fancy}
\renewcommand{\headrulewidth}{0pt}
\fancyhead{}
\rfoot{Page \thepage\  of \pageref{LastPage}}
\cfoot{}
\begin{document}
\title{Introduction to DCF Valuation and Cost of Capital Estimation}
\author{Nick DeRobertis}
\date{\today}
\maketitle
\begin{section}{Introduction to Discounted Cash Flow (DCF) Valuation}
\begin{itemize}
\item The Discounted Cash Flow (DCF) valuation of a stock is often considered a capstone finance model as incorporates many different concepts and also the technical skills to implement them
\item I have often heard of job applicants being asked to prepare a DCF model to prove their skills and knowledge
\item It is also generally considered the valuation approach which can be the most accurate, though there are a lot of assumptions which must be made which could influence the results
\item There are two main portions of the DCF model: coming up with the weighted average cost of capital (WACC) and estimating future free cash flows. Each of these portions have smaller tasks involved
\item At the end of the day, the concept of the model is extremely simple: take the present value of future cash flows to determine the value of the company. The difficult part is figuring out those cash flows and the discount rate
\item In this segment we will be focusing on the cost of capital estimation, and we will come back in the next segment to cover FCF estimation
\item We will focus on using the Capital Asset Pricing Model (CAPM) to estimate the cost of equity and we will discuss several approaches for estimating the cost and market value of debt depending on the availability of data and amount of time that can be invested into building the model
\end{itemize}
\end{section}
\begin{section}{Enterprise Value and Equity Value}
\begin{itemize}
\item When we take the present value of future cash flows, we will get the enterprise value which is a combination of the value from different sources of capital
\item Ultimately we are interested in determining the value of the stock which represents only equity value, and so we need to extract the equity value from the enterprise value by removing the other components
\item Many struggle with the concept that additional cash reduces enterprise value. I think it is useful to think through scenarios with the two interpretations of enterprise value: asset value or cost to acquire the company. In the context of cost to acquire, if an acquirer buys the business, they immediately own that cash on hand, so really the cash on hand offsets the purchase price. In the context of asset value, imagine two companies with identical operations. One decides to issue an additional \$10M of stock just to put cash on the balance sheet. Now they have \$10M of additional equity value, and if cash also added to enterprise value then all of a sudden they would be worth \$20M more than the other company despite the same operations. If the cash comes in the equation negatively, then there is no change in value for the stock issuance, which makes sense as the two companies still have the same operations
\end{itemize}
\end{section}
\begin{section}{Introduction to Cost of Equity}
\begin{itemize}
\item We are using the Capital Asset Pricing Model (CAPM) to estimate the cost of equity for the company
\item There are many other possible models which could be used such as one of the many factor models including Fama-French factors, but CAPM is the most basic and once you understand the approach it is not difficult to switch for a more complex model
\item In the general approach, we estimate the model on historical data and assume that the resulting estimated parameters will hold in the future to predict the future cost of capital
\end{itemize}
\end{section}
\begin{section}{Cost of Equity in Python}
\begin{itemize}
\item The process to estimating cost of capital using prior stock prices is straightforward: calculate returns, run the CAPM regression, then take the coefficient on the market risk premium (MRP) as the Beta, then plug the beta in the CAPM to estimate the future cost of equity based on an expected market return and risk free rate
\item Once you calculate returns, the first row will have missing data as there is no prior price. Statsmodels cannot handle missing data, so we will need to add a Pandas command to remove those rows
\end{itemize}
\end{section}
\begin{section}{Cost of Equity in Excel}
\begin{itemize}
\item The process to estimating cost of capital using historical prices in Excel is similar to doing it in Python
\item The main difference is just how we run the regression with the Data Analysis Tookpack GUI rather than through statsmodels
\item A disappointing aspect of doing this in Excel is it cannot automatically update if you add new data
\end{itemize}
\end{section}
\begin{section}{Market Value of Equity}
\begin{itemize}
\item If you are dealing with a publicly traded company, the calculation of market value of equity is simply number of shares multiplied price per share
\item With a private company, this is much more difficult. You can look at public comparables to help with this. If a public competitor has a \$10B valuation, and this company has 10\% of its market share and similar profit margins, a \$1B valuation might be a reasonable estimate
\item With early stage companies, you might not even have financials or a reasonable public comparable. But often these companies do not have debt and so it is not necessary to estimate the market value as you know it will be 100\% of the capitalization
\end{itemize}
\end{section}
\begin{section}{Cost of Debt}
\begin{itemize}
\item If your company is very mature and stable, the financial statements approach to cost of debt may be a reasonable approach
\item If your company does not have market bonds outstanding or you do not have access to data on these bonds, you are stuck with the financial statement approach regardless
\item If you can find information on even one market bond, it may be better to use the market rate of bonds approach, though this approach is even better when you can take a weighted average of multiple bonds
\item The handling of taxes is crucial in dealing with the cost of debt considering that debt is tax-advantaged in the US and many countries
\item The tax rate can be estimated from historical financial statements
\end{itemize}
\end{section}
\begin{section}{Introduction to Market Value of Debt}
\begin{itemize}
\item Simple models often just assume the market value of debt is equal to the book value of debt. While this makes the calculation simple and is sometimes all you can do due to data availability, this can be quite inaccurate for most companies
\item The market value of individual instruments approach can be the most accurate but requires the most data and takes more effort to implement
\item We are still keeping things a bit simple for both MV and cost of debt. In a more detailed model you would also consider seniority of the debt when doing these calculations
\end{itemize}
\end{section}
\begin{section}{Calculating the Market Value of Debt in Python}
\begin{itemize}
\item The value of a hypothetical bond approach just uses traditional bond valuation techniques
\item For the market value of individual bonds approach, again we are just applying traditional bond valuation, but we can use Pandas to apply the single calculation across all the bonds at once
\item We are also covering some material here on working with dates as you will usually have a maturity date to work with and need to convert it into a number of years
\end{itemize}
\end{section}
\begin{section}{Calculating the Weighted Average Cost of Capital (WACC)}
\begin{itemize}
\item The WACC is simply the weighted average of the costs of each source of capital
\item The weights are the percentage of the capital structure which is allocated to the type of asset, i.e. the market value of the source of capital divided by the total market value of the company
\item Be sure to use the after-tax cost of debt in the calculation
\item Companies with the same costs of capital can have very different WACCs due to a different percentage of debt and equity in the capital structure
\end{itemize}
\end{section}
\end{document}