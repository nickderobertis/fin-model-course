\documentclass[]{article}
\usepackage{booktabs}
\usepackage{amsmath}
\usepackage{pdflscape}
\usepackage{array}
\usepackage{threeparttable}
\usepackage{fancyhdr}
\usepackage{lastpage}
\usepackage{textcomp}
\usepackage{dcolumn}
\newcolumntype{L}[1]{>{\raggedright\let\newline\\\arraybackslash\hspace{0pt}}m{#1}}
\newcolumntype{C}[1]{>{\centering\let\newline\\\arraybackslash\hspace{0pt}}m{#1}}
\newcolumntype{R}[1]{>{\raggedleft\let\newline\\\arraybackslash\hspace{0pt}}m{#1}}
\newcolumntype{.}{D{.}{.}{-1}}
\usepackage[T1]{fontenc}
\usepackage{caption}
\usepackage{subcaption}
\usepackage{graphicx}
\usepackage[margin=0.8in, bottom=1.2in]{geometry}
\usepackage[page]{appendix}
\setcounter{secnumdepth}{0}
\pagestyle{fancy}
\renewcommand{\headrulewidth}{0pt}
\fancyhead{}
\rfoot{Page \thepage\  of \pageref{LastPage}}
\cfoot{}
\begin{document}
\title{Full DCF Valuation}
\author{Nick DeRobertis}
\date{\today}
\maketitle
\begin{section}{Overview}
\begin{subsection}{Problem Definition}
The purpose of this exercise is to complete a full discounted cash flow valuation of a stock from end to end,
complete with all of the additional analyses you learned throughout the course. You can pick any publicly traded
stock for your valuation. You must find the data on your own and research the company's operations. Ultimately the
main output is your valuation of the stock, but you must also provide a written justification of why you
believe this value to be correct. You must discuss and show how variable this estimate is, as well as what could
have large effects on the valuation. You should also consider several realistic scenarios based on states of the
economy, and how these scenarios affect the valuation.

Some of the components of your project should include: \begin{itemize}
\item WACC estimation
\item FCF estimation and forecasting (must forecast financial statements, not only FCFs directly, though that can be an extra check)
\item A written justification for why statements were forecasted as they were
\item Terminal value estimation using both perpetuity growth and various exit multiples
\item Monte carlo simulation
\item Sensitivity analysis
\item Scenario analysis
\item Visualization
\end{itemize}

\end{subsection}
\begin{subsection}{Notes}
Unlike the prior exercises, there is no solution key, as everyone will be working on a different company. Further,
there is considerable flexibility in forecasting so solutions wouldn't make sense even with the same company.

Readability is a higher portion of the grade as these will be large models and it's very important to stay
organized with a large model.
\end{subsection}
\begin{subsection}{Bonus}
Since this exercise is a lot less structured than prior projects, so too is the bonus. The bonus will be based on
doing parts of the model especially well. If your model stands out for having great visualization, that will earn
some bonus points. If it has a great write up, is organized very well, is especially thorough, etc. then you
can earn bonus points. Basically any part of the project, if you are doing it especially well you could earn
bonus points for that. The maximum on the bonus is 5\%.
\end{subsection}
\end{section}
\begin{section}{Submission \& Grading}
\begin{subsection}{Submission}
Please submit your model, a write up of the results, and any data you used. The write up can be included in the
model if desired.
\end{subsection}
\begin{subsection}{Grading}
\begin{center}
\begin{tabular}{l|c}
\multicolumn{2}{c}{Grading Breakdown}\\

\toprule
Category & Percentage\\

\cmidrule(lr){1-2}
Model Accuracy & 60\%\\
Model Readability & 30\%\\
Model Formatting & 10\%\\
Bonus & 5\%\\

\midrule
Total Possible & 105\%\\

\bottomrule
\end{tabular}
\end{center}
\end{subsection}
\end{section}
\end{document}