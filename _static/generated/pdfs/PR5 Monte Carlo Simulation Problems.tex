\documentclass[]{article}
\usepackage{booktabs}
\usepackage{amsmath}
\usepackage{pdflscape}
\usepackage{array}
\usepackage{threeparttable}
\usepackage{fancyhdr}
\usepackage{lastpage}
\usepackage{textcomp}
\usepackage{dcolumn}
\newcolumntype{L}[1]{>{\raggedright\let\newline\\\arraybackslash\hspace{0pt}}m{#1}}
\newcolumntype{C}[1]{>{\centering\let\newline\\\arraybackslash\hspace{0pt}}m{#1}}
\newcolumntype{R}[1]{>{\raggedleft\let\newline\\\arraybackslash\hspace{0pt}}m{#1}}
\newcolumntype{.}{D{.}{.}{-1}}
\usepackage[T1]{fontenc}
\usepackage{caption}
\usepackage{subcaption}
\usepackage{graphicx}
\usepackage[margin=0.8in, bottom=1.2in]{geometry}
\usepackage[page]{appendix}
\setcounter{secnumdepth}{0}
\pagestyle{fancy}
\renewcommand{\headrulewidth}{0pt}
\fancyhead{}
\rfoot{Page \thepage\  of \pageref{LastPage}}
\cfoot{}
\begin{document}
\title{Monte Carlo Simulation Problems}
\author{Nick DeRobertis}
\date{\today}
\maketitle
\begin{section}{Capital Budgeting Probabilities with Monte Carlo Simulation}
\begin{subsection}{Problem Definition}
You are a financial analyst for an aircraft manufacturer. Your company is trying to decide how large its next line of planes should be. Developing a line of planes takes a one-time research cost, then each plane has a cost to manufacture afterwards. The larger the plane, the higher the research and manufacture costs, but also the more revenue per plane. The larger planes are more risky because if the economy goes poorly, not many airlines will want to invest in such a large plane, but if the economy goes well, airlines will be rushing to the larger planes to fit rising demand.


                        The research cost will be paid at $t=0$. For simplicity, assume that all planes are manufactured 
                        at $t=1$ and sold at $t=2$. The interest rate is given in the below table.
                        
                        Find the expected NPV and standard deviation of NPV for each plane. Which plane has the lowest
                        chance of a negative NPV? Which has the highest chance of a positive NPV? Visualize the range
                        of possible NPVs for each plane, as well as the probability of acheiving different NPV levels 
                        (probability table). For the mid-size plane, which has a larger impact on the NPV, an 
                        additional plane sold or a decrease in the interest rate by 1\%? In your opinion, which
                        plane should the manufacturer create?
                        
\end{subsection}
\begin{subsection}{Possible Planes}
\begin{center}
\begin{tabular}{l|ccccc}
\toprule
Plane & Research Cost & Manufacture Cost & Sale Price & Expected Unit Sales & Stdev Unit Sales\\

\midrule
Super Size & \$100,000,000 & \$10,000,000 & \$11,500,000 & 200 & 120\\
Large & \$50,000,000 & \$5,000,000 & \$5,600,000 & 400 & 50\\
Mid-size & \$25,000,000 & \$3,000,000 & \$3,350,000 & 500 & 20\\

\bottomrule
\end{tabular}
\end{center}
\end{subsection}
\begin{subsection}{Other Inputs}
\begin{center}
\begin{tabular}{l|cc}
\toprule
Input & Default Value & Default Stdev\\

\midrule
Interest Rate & 7\% & 4\%\\

\bottomrule
\end{tabular}
\end{center}
\end{subsection}
\end{section}
\end{document}