\documentclass[]{article}
\usepackage{booktabs}
\usepackage{amsmath}
\usepackage{pdflscape}
\usepackage{array}
\usepackage{threeparttable}
\usepackage{fancyhdr}
\usepackage{lastpage}
\usepackage{textcomp}
\usepackage{dcolumn}
\newcolumntype{L}[1]{>{\raggedright\let\newline\\\arraybackslash\hspace{0pt}}m{#1}}
\newcolumntype{C}[1]{>{\centering\let\newline\\\arraybackslash\hspace{0pt}}m{#1}}
\newcolumntype{R}[1]{>{\raggedleft\let\newline\\\arraybackslash\hspace{0pt}}m{#1}}
\newcolumntype{.}{D{.}{.}{-1}}
\usepackage[T1]{fontenc}
\usepackage{caption}
\usepackage{subcaption}
\usepackage{graphicx}
\usepackage[margin=0.8in, bottom=1.2in]{geometry}
\usepackage[page]{appendix}
\setcounter{secnumdepth}{0}
\pagestyle{fancy}
\renewcommand{\headrulewidth}{0pt}
\fancyhead{}
\rfoot{Page \thepage\  of \pageref{LastPage}}
\cfoot{}
\begin{document}
\title{Python Retirement Savings Rate Problem}
\author{Nick DeRobertis}
\date{\today}
\maketitle
\begin{section}{Capital Budgeting Probabilities with Monte Carlo Simulation}
\begin{subsection}{Problem Definition}
We have already seen how we can build a retirement model in Python. The purpose of this exercise is to extend the
retirement model by making a more realistic assumption for the savings rate. To ease the complexity a bit and focus
only on the new part of the model, let's assume that the salary grows at a constant rate
\texttt{Salary Growth}, rather than working in promotions.

Now assume that the savings rate is dependent on the salary. If the salary is lower than
\texttt{Mid-Salary Cutoff}, then the individual can only save at a rate of
\texttt{Low Savings Rate}. If the salary is above that but less than \texttt{High-Salary Cutoff},
then the individual can save at \texttt{Mid Savings Rate}. If the salary is higher than
\texttt{High-Salary Cutoff}, the individual can save at \texttt{Mid Savings Rate}.

Determine the number of years to retirement for an individual with the default values of the inputs. Ensure that your
model is able to adjust for different inputs.

\end{subsection}
\begin{subsection}{Inputs}
\begin{center}
\begin{tabular}{l|cc}
\toprule
Input & Default Value\\

\midrule
Starting Salary & \$50,000\\
Salary Growth & 3\%\\
Mid-Salary Cutoff & \$80,000\\
High-Salary Cutoff & \$120,000\\
Low Savings Rate & 10\%\\
Mid Savings Rate & 25\%\\
High Savings Rate & 40\%\\
Interest Rate & 5\%\\
Desired Cash & \$1,500,000\\

\bottomrule
\end{tabular}
\end{center}
\end{subsection}
\begin{subsection}{Solution}

                        The final answer with the default inputs should be 37 years to retirement. Try hard to get
                        there working from scratch. If you are very stuck, then try taking the Dynamic Salary
                        Retirement model and modifying it. If you are still stuck, then check the provided Jupyter 
                        notebook solution. If you have a lot of trouble with this, please see me in office hours or
                        after class, as your first project will be similar but a bit more difficult.
                        
\end{subsection}
\end{section}
\end{document}