\documentclass[]{article}
\usepackage{amsmath}
\usepackage{pdflscape}
\usepackage{booktabs}
\usepackage{array}
\usepackage{threeparttable}
\usepackage{fancyhdr}
\usepackage{lastpage}
\usepackage{textcomp}
\usepackage{dcolumn}
\newcolumntype{L}[1]{>{\raggedright\let\newline\\\arraybackslash\hspace{0pt}}m{#1}}
\newcolumntype{C}[1]{>{\centering\let\newline\\\arraybackslash\hspace{0pt}}m{#1}}
\newcolumntype{R}[1]{>{\raggedleft\let\newline\\\arraybackslash\hspace{0pt}}m{#1}}
\newcolumntype{.}{D{.}{.}{-1}}
\usepackage[T1]{fontenc}
\usepackage{caption}
\usepackage{subcaption}
\usepackage{graphicx}
\usepackage[margin=0.8in, bottom=1.2in]{geometry}
\usepackage[page]{appendix}
\pagestyle{fancy}
\renewcommand{\headrulewidth}{0pt}
\fancyhead{}
\rfoot{Page \thepage\  of \pageref{LastPage}}
\cfoot{}
\begin{document}
\title{Exploring the Parameter Space}
\author{Nick DeRobertis}
\date{\today}
\maketitle
\begin{section}{Introduction to Parameter Exploration}
\begin{itemize}
\item If you've merely done a calculation, you have a single answer and can't learn any more about the problem without redoing the calculations
\item Instead because we are building models which can flexibly take any input and convert that into the appropriate output, it opens up a lot of possible extensions to the model
\item This is our main focus for a large portion of the course: how can we extend any financial model to get a greater understanding of the underlying problem
\item Sensitivity analysis is useful to understand how the full range of possible inputs affects the main results of your model
\item Scenario analysis allows analyzing outcomes in example situations
\item Monte Carlo Simulation allows assigning outputs to a probability distribution, which helps you understand the risk of your result
\end{itemize}
\end{section}
\begin{section}{Introduction to Sensitivity Analysis}
\begin{itemize}
\item The formal definition of sensitivity analysis may seem complicated, but all we are doing is running the model multiple times with different inputs and showing the outputs
\item A key component of sensitivity analysis is the visualization, as now that there are many different outputs it is much easier to draw meaning from them when visualized
\end{itemize}
\end{section}
\begin{section}{Sensitivity Analysis in Excel}
\begin{itemize}
\item Data tables in Excel allow calculating a cell multiple times, changing some other cell. This is perfect for sensitivity analysis if we target an output cell and change an input cell
\item One-way data tables change one input at a time, two-way data tables change two inputs at a time
\item You are basically limited to changing two inputs at once, without doing some clever hacks
\item Visualization rule of thumb: graph one-way data tables and use conditional formatting for two-way
\item Conditional formatting changes the format of cells based on conditions, such as putting the largest numbers in green and the smallest in red
\item Row input cell means that your inputs are going horizontally in a row. Column input cell means that your inputs going vertically in a column. For one-way data tables, you will use only one of the two. For two-way data tables, you will use both
\end{itemize}
\end{section}
\begin{section}{Using Python Dictionaries}
\begin{itemize}
\item There are three ways to loop through dictionaries: through the keys (the default), through the values (.values()), and through both at once (.items())
\item If you want one way to do it, just always loop through the items as you have access to both the key and the value at once
\item Combine dictionaries using update. Add items to dictionaries with brackets and assignment. Remove items from dictionaries with pop
\end{itemize}
\end{section}
\begin{section}{Python List Comprehensions - Convenient List Building}
\begin{itemize}
\item List comprehensions are an example of "syntactic sugar," or a feature of a programming language which is not necessary but makes things easier (makes the programming experience "sweeter")
\item They allow us to write simple loops made to create lists with only a single line of code
\end{itemize}
\end{section}
\begin{section}{Python Imports and Installing Packages}
\begin{itemize}
\item Import can be used for third-party and built-in packages, but also for your own code offloaded into separate files
\item I had you install Anaconda to get Python because it includes most of the Python packages we would want to use pre-installed. So we haven't had to install any package up to now
\item With more than 250k packages out there and only about 200 installed in Anaconda, the time will come when you need to install something
\item It will even happen in this course because we will use some packages I have created
\end{itemize}
\end{section}
\begin{section}{Introduction to Sensitivity Analysis in Python}

\end{section}
\begin{section}{Sensitivity Analysis in Python Example}

\end{section}
\end{document}